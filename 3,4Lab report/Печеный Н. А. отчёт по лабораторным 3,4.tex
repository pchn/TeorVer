\documentclass[12pt]{article}
\usepackage[utf8]{inputenc}
\usepackage[russian]{babel}
\usepackage{pscyr}
\usepackage[T2A]{fontenc}
\usepackage{geometry}
\usepackage{amsmath}
\usepackage{amssymb}
\usepackage{graphicx}
%\usepackage{listings}
\usepackage{xcolor}
\usepackage{multirow}
\usepackage[unicode, pdftex]{hyperref}
\DeclareUnicodeCharacter{2212}{-}
\geometry {	
	a4paper, 
	left   = 20mm, 
	right  = 20mm, 
	top    = 20mm, 
	bottom = 20mm
}

\definecolor{urlcolor}{HTML}{000000} 
\definecolor{linkcolor}{HTML}{000000}

\graphicspath{{resource/}}



\newcommand{\lskip}{\hfill\break}


\begin{document}

\begin{titlepage}
	\begin{center}
		\hfill \break
		{\textbf{Санкт-Петербургский политехнический университет Петра Великого}}\\
		\hfill \break
		\textbf{Институт прикладной математики и механики}\\
		 \hfill \break
		\textbf{Кафедра <<Телематика (при ЦНИИ РТК)>>}\\
		\vfill
		\large{\bfseries Отчет по лабораторным работам № 3, 4}\\
		\hfill \break
		\hfill \break
		\hfill \break
		\hfill \break
		По дисциплине <<Теория вероятностей и Математическая статистика>>\\
		\hfill \break
		\hfill \break
		\hfill \break
	\end{center}
 
	\normalsize
	{ 
		\begin{tabular}{lp{2cm}cr}
			Выполнил &&&\\
			Студент гр. 3630201/80101&&\underline{\hspace{1.5cm}}& Печеный Н. А.\\\\
			Руководитель&&&\\ 
			к.ф.-м.н., доцент && \underline{\hspace{1.5cm}}& Баженов А. Н. \\\\
			&&&<<\underline{\phantom{333}}>>\underline{\phantom{сентября000}}
			2020г.
		\end{tabular}
	}
\vfill

\begin{center} Санкт-Петербург \\2020 \end{center}
\end{titlepage}

\newpage

\setcounter{page}{2}

\setlength{\parindent}{1cm}

\tableofcontents

\newpage

\listoffigures

\newpage

\listoftables

\newpage

\section{Постановка задачи}
Для 5 распределений:
\begin{itemize}
    \item Нормальное распределение $N(x, 0, 1)$
    \item Распределение Коши $C(x, 0, 1)$
    \item Распределение Лапласа $L(x, 0, 1/\sqrt{2})$
    \item Распределение Пуассона $P(k, 10)$
    \item Равномерное распределение $U(x, -\sqrt{3}, \sqrt{3})$
\end{itemize}
Требуется:
\begin{enumerate}
    \item Сгенерировать выборки размером 20 и 100 элементов.
    Построить для них боксплот Тьюки. Для каждого распределения определить долю выбросов экспериментально (сгенерировав выборку, соответствующую распределению 1000 раз, и вычислив среднюю долю выбросов) и сравнить с результатами, полученными теоретически.
    \item Сгенерировать выборки размером 20, 60 и 100 элементов. Построить на них эмпирические функции распределения и ядерные оценки плотности распределения на отрезке [−4;4] для непрерывных распределений и на отрезке [6;14] для распределения Пуассона.
\end{enumerate}

\newpage

\section{Теория}
\subsection{Боксплот Тьюки}
\subsubsection{Построение}
Границы ящика — первый и третий квартили, линия в середине ящика — медиана. Концы усов — края статистически значимой выборки (без выбросов). Длина «усов»:
\begin{equation}
    X_1 = Q_1 - \frac{3}{2}(Q_3-Q_1),\; 
    X_2 = Q_3 + \frac{3}{2}(Q_3-Q_1)
    \label{x1x2}
\end{equation}
где $X_1$ — нижняя граница уса, $X_2$ — верхняя граница уса, $Q_1$ — первый
квартиль, $Q_3$ — третий квартиль.\\
Данные, выходящие за границы усов (выбросы), отображаются на графике
в виде маленьких кружков \cite{Box_plot}.\\

\subsection{Теоретическая вероятность выбросов}
Можно вычислить теоретические первый и третий квартили распределений
--- $Q_1^T$ и $Q_3^T$. По ф-ле (\ref{x1x2}) --- теоретические нижнюю и верхнюю границы уса --- $X_1^T$ и $X_2^T$. Выбросы --- величины $x$:
\[
	\left[
		\begin{aligned}
			& x < X_1^T\\
			& x > X_2^T\\
		\end{aligned}
	\right.
\]
Теоретическая вероятность выбросов:
\begin{itemize}
	\item для непрерывных распределений
	\begin{equation}
		P^T_B = P(x < X_1^T) + P(x > X_2^T) = F(X_1^T) + (1 - F(X_2^T))
		\label{continuousP}
	\end{equation}
	\item для дискретных распределений
	\begin{equation}
		P^T_B = P(x < X_1^T) + P(x > X_2^T) = (F(X^T_1) - P(X = X^T_1)) + (1-F(X^T_2))
		\label{discreteP}
	\end{equation}
\end{itemize}
где $F(X) = P(x \leqslant X)$ --- функция распределения.
\subsection{Эмпирическая функция распределения}
\subsubsection{Статистический ряд}
Статистический ряд --- последовательность различных элементов выборки $z_1, z_2, \dots, z_k$, расположенных в возрастающем порядке с указанием частот $n_1, n_2, \dots, n_k$, с которыми эти элементы содержатся в выборке. Обычно записывается в виде таблицы \cite{theory}.
\subsubsection{Эмпирическая функция распределения}
Эмпирическая (выборочная) функция распределения (э.~ф.~р.) --- относительная частота события $X < x$, полученная по данной выборке:
\[
	X_n^*(x) = P^*(X < x).
\]
\subsubsection{Нахождение э.~ф.~р.}
Для получения относительной частоты $P^*(X < x)$ просуммируем в статистическом ряде, построенном по данной выборке, все частоты $n_i$, для которых элементы $z_i$ статистического ряда меньше $x$. Тогда $P^*(X < x) = \frac{1}{n}\sum\limits_{z_i < x}n_i$. Получаем
\[
	F^*(x) = \frac{1}{n}\sum_{z_i < x}n_i
\]
$F^*(x)$ --- функция распределения дискретной случайной величины $X^*$, заданной таблицей распределения, приведённой ниже в Таблице 1.
\begin{table}[h!]
    \caption{Таблица распределения}
    \begin{center}
    \begin{tabular}{|c|c|c|c|c|}
    \hline
    $X^*$&$z_1$&$z_2$&$\dots$&$z_k$\\
    \hline
	$P$&$\frac{n_1\mathstrut}{n\mathstrut}$&$\frac{n_2}{n}$&$\dots$&$\frac{n_k}{n}$\\
    \hline
    \end{tabular}
\end{center}
\end{table}

\noindentЭмпирическая функция распределения является оценкой, т.~е. приближённым значением, генеральной функции распределения
\begin{equation}
	F_n^*(x) \approx F_X(x).
	\label{empiric}
\end{equation}
\subsection{Оценки плотности вероятности}
\subsubsection{Определение}
Оценкой плотности вероятности $f(X)$ называется функция $\hat f(x)$, построенная на основе выборки, приближённо равная $f(x)$
\[
	\hat f(x) \approx f(x)
\]
\subsubsection{Ядерные оценки}
Представим оценку в виде суммы с числом слагаемых, равным объёму выборки:
\[
	\hat{f_n}(x) = \frac{1\mathstrut}{nh_n\mathstrut} \sum_{i=1}^{n}K\left(\frac{x-x_i\mathstrut}{h_n\mathstrut}\right)
\]
Здесь функция $K(u)$, называемая ядерной (ядром), непрерывна и является плотностью вероятности, $x_1, \dots, x_n$ --- элементы выборки, $\left\{h_n\right\}$ --- любая последовательность положительных чисел, облажающая свойствами
\[
	h_n \xrightarrow[n \to \infty]{} 0;\qquad \frac{h_n\mathstrut}{n^{-1}\mathstrut}\xrightarrow[n \to \infty]{} \infty.
\]
Такие оценки называются непрерывными ядерными.\\
Гауссово (нормальное) ядро:
\begin{equation}
	K(u) = \frac{1\mathstrut}{\sqrt{2\,\pi}\mathstrut}e^{-\frac{u^2\mathstrut}{2\mathstrut}}.
	\label{gaussian_kernel}
\end{equation}
Правило Сильвермана:
\begin{equation}
	h_n = 1.06\hat\sigma n^{-1/5},
	\label{silverman}
\end{equation}
где $\hat\sigma$ --- выборочное стандартное отклонение\cite{theory}.
\newpage
\section{Реализация}
Расчёты проводились в среде аналитических вычислений Maxima. Для генерации выборок и создания и отрисовки графиков были использованы библиотечные функции среды разработки. Код скрипта представлен в репозитории на GitHub, ссылка на репозиторий находится в \hyperlink{addition}{\textbf{Приложении А}}.
\newpage
\section{Результаты}
\subsection{Боксплот Тьюки}
На каждом из представленных рисунков 1-5 по горизонатльной оси отложены значения элементов выборки, по вертикальной - размеры выборок.\\
\begin{figure}[h]
	\center{\includegraphics[width=0.75\linewidth]{Ris1.png}}
	\caption{Боксплоты выборок нормального распределения}
\end{figure}
\begin{figure}[h]
	\center{\includegraphics[width=0.75\linewidth]{Ris2.png}}
	\caption{Боксплоты выборок распределения Коши}
\end{figure}
\newpage
\begin{figure}[h]
	\center{\includegraphics[width=0.75\linewidth]{Ris3.png}}
	\caption{Боксплоты выборок распределения Лапласа}
\end{figure}
\begin{figure}[h]
	\center{\includegraphics[width=0.75\linewidth]{Ris4.png}}
	\caption{Боксплоты выборок распределения Пуассона}
\end{figure}
\newpage
\begin{figure}[h]
	\center{\includegraphics[width=0.75\linewidth]{Ris5.png}}
	\caption{Боксплоты выборок равномерного распределения}
\end{figure}
\subsection{Сравнение теоретической вероятности и экспериментальной доли выбросов}
Для каждого распределения были 1000 раз сгенерированы выборки, соответствующие распределению, была вычислена средня доля выбросов. Погрешность средней доли выбросов была рассчитана по формуле
\begin{equation}
	\Delta_z = \sqrt{\left(\overline{z^2} - {\overline{z}}^2\right)}
    \label{delta}
\end{equation}
Значения экспериментальной средней доли выбросов были округлены в соответствии с погрешностями. Результаты представлены ниже в Таблице 2.
\begin{table}[h]
	\caption{Сравнение экспериментальной доли и теоретической вероятности выбросов}
	\begin{center}
		\begin{tabular}{|*{4}{c|}}\hline
			\multirow{2}*{Распределение} & \multirow{2}*{$N$} & Доля & \multirow{2}*{$P_{B}^{T}$(\ref{continuousP})}  \\
			& & выбросов &\\ \hline
			\multirow{2}*{Нормальное} & 20 &$0.00 \pm 0.02\mathstrut$ & \multirow{2}*{0.0069}\\ \cline{2-3}
			& 100 & $ 0.007 \pm 0.008\mathstrut$ & \\ \hline
			\multirow{2}*{Коши} & 20 &$0.16 \pm 0.08\mathstrut$ & \multirow{2}*{0.156}\\ \cline{2-3}
			& 100 & $ 0.16 \pm 0.04\mathstrut$ & \\ \hline
			\multirow{2}*{Лапласа} & 20 &$0.06 \pm 0.06\mathstrut$ & \multirow{2}*{0.0625}\\ \cline{2-3}
			& 100 & $ 0.06 \pm 0.02\mathstrut$ & \\ \hline
			\multirow{2}*{Пуассона} & 20 &$0.01 \pm 0.02\mathstrut$ & \multirow{2}*{0.0099}\\ \cline{2-3}
			& 100 & $ 0.008 \pm 0.009\mathstrut$ & \\ \hline
			\multirow{2}*{Нормальное} & 20 &$0.0 \pm 0.0\mathstrut$ & \multirow{2}*{0.0}\\ \cline{2-3}
			& 100 & $ 0.0 \pm 0.0\mathstrut$ & \\ \hline
		\end{tabular}
	\end{center}
\end{table}

\subsection{Эмпирическая функция респределения}
На рисунках 6-10 представлены графики, по которым можно оценить отклонение эмпирических функций (\ref{empiric}) распределений от теоретических.
\begin{figure}[h!]
	\begin{center}
	\begin{minipage}[h]{0.60\linewidth}
		\center{\includegraphics[width=1\linewidth]{Ris11.png}}
	\end{minipage}
	\phantom{0}\\
	\begin{minipage}[h]{0.60\linewidth}
		\center{\includegraphics[width=1\linewidth]{Ris12.png}}
	\end{minipage}
	\phantom{0}\\
	\begin{minipage}[h]{0.60\linewidth}
		\center{\includegraphics[width=1\linewidth]{Ris13.png}}
	\end{minipage}
	\caption{Э.~ф.~р. нормального распределения $N(x, 0, 1)$}
\end{center}
\end{figure}
\newpage
\begin{figure}[h!]
	\begin{center}
	\begin{minipage}[h]{0.65\linewidth}
		\center{\includegraphics[width=1\linewidth]{Ris14.png}}
	\end{minipage}
	\phantom{0}\\
	\begin{minipage}[h]{0.65\linewidth}
		\center{\includegraphics[width=1\linewidth]{Ris15.png}}
	\end{minipage}
	\phantom{0}\\
	\begin{minipage}[h]{0.65\linewidth}
		\center{\includegraphics[width=1\linewidth]{Ris16.png}}
	\end{minipage}
	\caption{Э.~ф.~р. распределения Коши $C(x, 0, 1)$}
\end{center}
\end{figure}
\newpage\begin{figure}[h!]
	\begin{center}
	\begin{minipage}[h]{0.65\linewidth}
		\center{\includegraphics[width=1\linewidth]{Ris17.png}}
	\end{minipage}
	\phantom{0}\\
	\begin{minipage}[h]{0.65\linewidth}
		\center{\includegraphics[width=1\linewidth]{Ris18.png}}
	\end{minipage}
	\phantom{0}\\
	\begin{minipage}[h]{0.65\linewidth}
		\center{\includegraphics[width=1\linewidth]{Ris19.png}}
	\end{minipage}
	\caption{Э.~ф.~р. распределения Лапласа $L(x, 0, 1/\sqrt{2})$}
\end{center}
\end{figure}
\newpage\begin{figure}[h!]
	\begin{center}
	\begin{minipage}[h]{0.65\linewidth}
		\center{\includegraphics[width=1\linewidth]{Ris110.png}}
	\end{minipage}
	\phantom{0}\\
	\begin{minipage}[h]{0.65\linewidth}
		\center{\includegraphics[width=1\linewidth]{Ris111.png}}
	\end{minipage}
	\phantom{0}\\
	\begin{minipage}[h]{0.65\linewidth}
		\center{\includegraphics[width=1\linewidth]{Ris112.png}}
	\end{minipage}
	\caption{Э.~ф.~р. распределения Пуассона $P(k, 10)$}
\end{center}
\end{figure}
\newpage\begin{figure}[h!]
	\begin{center}
	\begin{minipage}[h]{0.65\linewidth}
		\center{\includegraphics[width=1\linewidth]{Ris113.png}}
	\end{minipage}
	\phantom{0}\\
	\begin{minipage}[h]{0.65\linewidth}
		\center{\includegraphics[width=1\linewidth]{Ris114.png}}
	\end{minipage}
	\phantom{0}\\
	\begin{minipage}[h]{0.65\linewidth}
		\center{\includegraphics[width=1\linewidth]{Ris115.png}}
	\end{minipage}
	\caption{Э.~ф.~р. равномерного распределения $U(x, -\sqrt{3}, \sqrt{3})$}
\end{center}
\end{figure}
\newpage


\subsection{Ядерные оценки}
На рисунках 11-15 представлены графики ядерных оценок плотности для выборок, соответствующих заданным распределениям. Ядерная функция имеет вид (\ref{gaussian_kernel}). Чёрной линией обозначена теоретическая плотность вероятности, красной линией обозначена ядерная оценка с параметром сглаживания $h_n/2$, зелёной --- $h_n$, синей --- $2h_n$. Параметр сглаживания рассчитан по формуле (\ref{silverman}).

\begin{figure}[h!]
	\begin{center}
	\begin{minipage}[h]{0.6\linewidth}
		\center{\includegraphics[width=1\linewidth]{Ris21.png}}
	\end{minipage}
	\phantom{0}\\
	\begin{minipage}[h]{0.6\linewidth}
		\center{\includegraphics[width=1\linewidth]{Ris22.png}}
	\end{minipage}
	\phantom{0}\\
	\begin{minipage}[h]{0.6\linewidth}
		\center{\includegraphics[width=1\linewidth]{Ris23.png}}
	\end{minipage}
	\caption{Ядерная оценка плотности нормального распределения $N(x, 0, 1)$}
\end{center}
\end{figure}
\newpage
\begin{figure}[h!]
	\begin{center}
	\begin{minipage}[h]{0.65\linewidth}
		\center{\includegraphics[width=1\linewidth]{Ris24.png}}
	\end{minipage}
	\phantom{0}\\
	\begin{minipage}[h]{0.65\linewidth}
		\center{\includegraphics[width=1\linewidth]{Ris25.png}}
	\end{minipage}
	\phantom{0}\\
	\begin{minipage}[h]{0.65\linewidth}
		\center{\includegraphics[width=1\linewidth]{Ris26.png}}
	\end{minipage}
	\caption{Ядерная оценка плотности распределения Коши $C(x, 0, 1)$}
\end{center}
\end{figure}
\newpage\begin{figure}[h!]
	\begin{center}
	\begin{minipage}[h]{0.65\linewidth}
		\center{\includegraphics[width=1\linewidth]{Ris27.png}}
	\end{minipage}
	\phantom{0}\\
	\begin{minipage}[h]{0.65\linewidth}
		\center{\includegraphics[width=1\linewidth]{Ris28.png}}
	\end{minipage}
	\phantom{0}\\
	\begin{minipage}[h]{0.65\linewidth}
		\center{\includegraphics[width=1\linewidth]{Ris29.png}}
	\end{minipage}
	\caption{Ядерная оценка плотности распределения Лапласа $L(x, 0, 1/\sqrt{2})$}
\end{center}
\end{figure}
\newpage\begin{figure}[h!]
	\begin{center}
	\begin{minipage}[h]{0.65\linewidth}
		\center{\includegraphics[width=1\linewidth]{Ris210.png}}
	\end{minipage}
	\phantom{0}\\
	\begin{minipage}[h]{0.65\linewidth}
		\center{\includegraphics[width=1\linewidth]{Ris211.png}}
	\end{minipage}
	\phantom{0}\\
	\begin{minipage}[h]{0.65\linewidth}
		\center{\includegraphics[width=1\linewidth]{Ris212.png}}
	\end{minipage}
	\caption{Ядерная оценка плотности распределения Пуассона $P(k, 10)$}
\end{center}
\end{figure}
\newpage\begin{figure}[h!]
	\begin{center}
	\begin{minipage}[h]{0.65\linewidth}
		\center{\includegraphics[width=1\linewidth]{Ris213.png}}
	\end{minipage}
	\phantom{0}\\
	\begin{minipage}[h]{0.65\linewidth}
		\center{\includegraphics[width=1\linewidth]{Ris214.png}}
	\end{minipage}
	\phantom{0}\\
	\begin{minipage}[h]{0.65\linewidth}
		\center{\includegraphics[width=1\linewidth]{Ris215.png}}
	\end{minipage}
	\caption{Ядерная оценка плотности равномерного распределения $U(x, -\sqrt{3}, \sqrt{3})$}
\end{center}
\end{figure}
\newpage
\section{Заключение}
\subsection{Экспериментальная доля и теоретическая вероятность выбросов}
В ходе выполнения лабораторной работы были построены боксплоты Тьюки для выборок разного размера, соответствующих заданным распределениям. По построенным боксплотам удалось визуально оценить мощность выбросов соответствующих распределений и сделать вывод о том, что наиболее подвержены выбросам выборки, построенные по распределению Коши.\\
Также были рассчитаны теоретические вероятности выбросов для каждого распределения. После этого на основе генерации 1000 выборок размерами в 20 и 100 элементов для каждого распределения были вычислены экспериментальные доли выбросов. На основании полученных результатов можно сделать вывод о том, что экспериментальные доли выбросов достаточно близки к теоретическим вероятностям и тем сильнее к ним приближаются, чем больше размер выборки.

\subsection{Эмпирическая функция и ядерные оценки плотности распределения}
В ходе выполнения лабораторной работы были построены эмпирические функции распределения для выборок разного размера, соответствующих пяти заданыым распределениям. Из построенных графиков можно сделать вывод, что эмпирическая функция распределения тем ближе к теоретической, чем больше размер выборки.\\
Также были построены ядерные оценки плотности вероятностей распределений. Из графиков можно сделать вывод, что чем шире полоса пропускания (больше параметр сглаживания), тем более более график сглажен и менее чувствителен к выбросам. Для нормального распределения и распределения Пуассона больше всего подходит выбор параметра сглаживания $h_n$, для равномерного распределения --- $2h_n$, для распределения Лапласа --- $h_n/2$. Для распределения Коши не подходит ни один из параметров сглаживания, при любом его выборе ядерная оценка достаточно далека от теоретической плотности вероятности.
\newpage
\addcontentsline{toc}{section}{Список Литературы}
\begin{thebibliography}{9}

	\bibitem{Box_plot}
	Box plot \url{https://en.wikipedia.org/wiki/Box_plot} Дата обращения 7.12.2020

	\bibitem{theory}
	Теоретическое приложение к лабораторным работам №1-4 по дисциплине «Математическая статистика». -- СПб.: СПбПУ, 2020. -- 12 c 

\end{thebibliography}
\newpage
\section*{\hypertarget{addition}{Приложение А. Репозиторий с исходным кодом}}
\addcontentsline{toc}{section}{Приложение А. Репозиторий с исходным кодом}
Ссылка на репозиторий GitHub с исходным кодом: \url{https://github.com/pchn/TeorVer}
\end{document}