\documentclass[12pt]{article}
\usepackage[utf8]{inputenc}
\usepackage[russian]{babel}
\usepackage{pscyr}
\usepackage[T2A]{fontenc}
\usepackage{geometry}
\usepackage{graphicx}
\usepackage{multirow}
%\usepackage{hhline}
\usepackage{amsmath}
\usepackage{amssymb}
\usepackage[unicode, pdftex]{hyperref}
\usepackage{xcolor}

\DeclareUnicodeCharacter{2212}{}
\geometry {	
	a4paper, 
	left   = 20mm, 
	right  = 20mm, 
	top    = 20mm, 
	bottom = 20mm
}

\definecolor{urlcolor}{HTML}{000000} 
\definecolor{linkcolor}{HTML}{000000}

\graphicspath{{resource/}}

\begin{document}

\begin{titlepage}
	\begin{center}
		\hfill \break
		{\textbf{Санкт-Петербургский политехнический университет Петра Великого}}\\
		\hfill \break
		\textbf{Институт прикладной математики и механики}\\
		\hfill \break
		\textbf{Кафедра <<Телематика (при ЦНИИ РТК)>>}\\
		\vfill
		\large{\bfseries Отчет по лабораторным работам № 5, 6}\\
		\hfill \break
		\hfill \break
		\hfill \break
		\hfill \break
		По дисциплине <<Теория вероятностей и Математическая статистика>>\\
		\hfill \break
		\hfill \break
		\hfill \break
	\end{center}
 
	\normalsize
	{ 
		\begin{tabular}{lp{2cm}cr}
			Выполнил &&&\\
			Студент гр. 3630201/80101&&\underline{\hspace{1.5cm}}& Печеный Н. А.\\\\
			Руководитель&&&\\ 
			к.ф.-м.н., доцент && \underline{\hspace{1.5cm}}& Баженов А. Н. \\\\
			&&&<<\underline{\phantom{333}}>>\underline{\phantom{сентября000}}
			2021г.
		\end{tabular}
	}
\vfill

\begin{center} Санкт-Петербург \\2021 \end{center}
\end{titlepage}

\newpage

\setcounter{page}{2}

\setlength{\parindent}{1cm}

\tableofcontents

\newpage

\listoffigures

\newpage

\listoftables

\newpage

\section{Постановка задачи}
\begin{enumerate}
    \item Сгенерировать двумерные выборки размерами 20, 60, 100 для нормального двумерного распределения $N(x,y,0,0,1,1,\rho)$ Коэффициент корреляции $\rho$ взять равным 0, 0.5, 0.9. Каждая выборка генерируется 1000 раз и для неё вычисляются: среднее значение, среднее значение квадрата и дисперсия коэффициентов корреляции Пирсона, Спирмена и квадрантного коэффициента корреляции. Повторить все вычисления для смеси нормальных распределений:
    \begin{equation}
        f(x,y) = 0.9N(x,y,0,0,1,1,0.9) + 0.1N(x,y,0,0,10,10,−0.9).
        \label{mix}
    \end{equation}
    Изобразить сгенерированные точки на плоскости и нарисовать эллипс
    равновероятности.
    \item Найти оценки коэффициентов линейной регрессии $y_i = a + bx_i + e_i$, используя 20 точек на отрезке [-1.8; 2] с равномерным шагом равным 0.2. Ошибку $e_i$ считать нормально распределённой с параметрами (0, 1). В качестве эталонной зависимости взять $y_i = 2 + 2x_i + e_i$ . При построении оценок коэффициентов использовать два критерия: критерий наименьших квадратов и критерий наименьших модулей. Проделать то же самое для выборки, у которой в значения $y_1$ и $y_20$ вносятся возмущения 10 и -10.
\end{enumerate}

\newpage

\section{Теория}

\subsection{Двумерное нормальное распределение}
Двумерная случайная величина $(X, Y)$ называется распределённой нормально (или просто нормальной), если её плотность вероятности определена формулой
\begin{multline}
    N(x, y, \overline{x}, \overline{y}, \sigma_x, \sigma_y, \rho) = \frac{1}{2\pi\sigma_x\sigma_y\sqrt{1 - \rho^2}} \times\\
    \times \exp\left\{ -\frac{1}{2(1 - \rho^2)} \left[ \frac{(x - \overline{x})^2}{\sigma_x^2} - 2\rho\frac{(x - \overline{x})(y - \overline{y})}{\sigma_x\sigma_y} + \frac{(y - \overline{y})^2}{\sigma_y^2}\right] \right\}
    \label{normal2}
\end{multline}
Компоненты $X, Y$ двумерной нормальной случайной величины также распределены нормально с математическими ожиданиями $\overline{x}, \overline{y}$ и средними квадратическими отклонениями $\sigma_x, \sigma_y$ соответственно \cite{theory}. Параметр $\rho$ называется коэффициентом корреляции.

\subsection{Корреляционный момент (ковариация) и коэффициент корреляции}
Корреляционный момент, иначе ковариация, двух случайных величин $X$ и $Y$:
    \begin{equation}
        K = cov(X, Y) = M\left[(X-\overline{x})(Y-\overline{y})\right]
    \end{equation}
\noindentКоэффициент корреляции $\rho$ двух случайных величин $X$ и $Y$:
    \begin{equation}
        \rho = \frac{K}{\sigma_x\sigma_y}
    \end{equation}

\subsection{Выборочные коэффициенты корреляции}
    
    \subsubsection{Выборочный коэффициент корреляции Пирсона}
    Выборочный коэффициент корреляции Пирсона \cite{theory}:
    \begin{equation}
        r = \frac{\frac{1}{n} \sum (x_i - \overline{x})(y_i - \overline{y})}{\sqrt{\frac{1}{n} \sum (x_i - \overline{x})^2 \frac{1}{n} \sum (y_i - \overline{y})^2}} = \frac{K}{s_Xs_Y},
    \label{pierson}
    \end{equation}
    где $K, s_X^2, s_Y^2$ -- выборочные ковариация и дисперсии с.в. $X$ и $Y$.

    \subsubsection{Выборочный квадрантный коэффициент корреляции}
    Выборочный квадрантный коэффициент корреляции \cite{theory}:
        \begin{equation}
        r_Q = \frac{(n_1+n_3) - (n_2 + n_4)}{n}, 
        \label{quadrant}
        \end{equation} 
    где $n_1, n_2, n_3, n_4$ -- количества точек с координатами $(x_i, y_i)$, попавшими соответственно в I, II, III и IV квадранты декартовой системы с осями $x' = x - med \; x, y' = y - med \; y$ и с центром в точке с координатами $(med \; x, med \; y)$.

    \subsubsection{Выборочный коэффициент ранговой корреляции Спирмена}
    Обозначим ранги, соответствующие значениям переменной $X$, через $u$, а ранги, соответствующие значениям переменной $Y$, — через $v$.\\
    Выборочный коэффициент ранговой корреляции Спирмена \cite{theory}:
    \begin{equation}
        r_S = \frac{\frac{1}{n} \sum (u_i - \overline{u})(v_i - \overline{v})}{\sqrt{\frac{1}{n} \sum (u_i - \overline{u})^2 \frac{1}{n} \sum (v_i - \overline{v})^2}},
        \label{spirmen}
    \end{equation} 
    где $\overline{u} = \overline{v} = \frac{1 + 2 + ... + n}{n} = \frac{n+1}{2}$ -- среднее значение рангов.

\subsection{Эллипсы рассеивания}
Уравнение проекции эллипса рассеивания на плоскость $xOy$:
\begin{equation}
    \frac{(x-\overline{x})^2}{\sigma_x^2} - 2\rho\frac{(x-\overline{x})(y-\overline{y})}{\sigma_x\sigma_y} + \frac{(y-\overline{y})^2}{\sigma_y^2} = const.
    \label{ellipse}
\end{equation}
Центр эллипса (\ref{ellipse}) находится в точке с координатами $(\overline{x}, \overline{y})$; оси симметрии эллипса составляют с осью $Ox$ углы, определяемые уравнением
\begin{equation}
    tg 2\alpha = \frac{2\rho\sigma_x\sigma_y}{\sigma_x^2 - \sigma_y^2}.
\end{equation} 

\subsection{Простая линейная регрессия}

    \subsubsection{Модель простой линейной регрессии}
    Регрессионную модель описания данных называют простой линейной регрессией, если
    \begin{equation}
        y_i = \beta_0 + \beta_1x_i + \varepsilon_i, \quad i=1, \dots, n,
    \end{equation}
    где $x_1, \dots, x_n$ — заданные числа (значения фактора); $y_1, \dots, y_n$ — наблюдаемые значения отклика; $\varepsilon_1, \dots, \varepsilon_n$ — независимые, нормально распределённые $N(0,\sigma)$ с нулевым математическим ожиданием и одинаковой (неизвестной) дисперсией случайные величины (ненаблюдаемые); $\beta_0, \beta_1$ — неизвестные параметры, подлежащие оцениванию.

    \subsubsection{Метод наименьших квадратов}
    Метод наименьших квадратов (МНК) \cite{theory}:
    \begin{equation}
        Q(\beta_0, \beta_1) = \sum_{i = 1}^{n}\varepsilon_{i}^{2} = \sum_{i = 1}^{n}(y_i - \beta_0 - \beta_1x_i)^2 \rightarrow \min_{\beta_0, \beta_1}.
    \end{equation}

    \subsubsection{Расчётные формулы для МНК-оценок}
    МНК-оценки параметров $\beta_0$ и $\beta_1$ \cite{theory}:
    \begin{equation}
        \hat{\beta}_1 = \frac{\overline{xy} - \overline{x} \cdot \overline{y}}{\overline{x^2} - (\overline{x})^2}
        \label{mnk1}
    \end{equation}
    \begin{equation}
        \hat{\beta}_0 = \overline{y} - \overline{x}\hat{\beta}_1
        \label{mnk0}
    \end{equation}

\subsection{Робастные оценки коэффициентов линейной регрессии}
Метод наименьших модулей \cite{theory}:
\begin{equation}
    \sum_{i = 1}^{n}|y_i - \beta_0 - \beta_1x_i| \rightarrow \min_{\beta_0, \beta_1}.
\end{equation}
\begin{equation}
    \hat{\beta}_{1R} = r_Q\frac{q^{*}_{y}}{q^{*}_{x}},
    \label{mnm1}
\end{equation}
\begin{equation}
    \hat{\beta}_{0R} = med \; y - \hat{\beta}_{1R} med \; x,
    \label{mnm0}
\end{equation}
\begin{equation}
    r_Q = \frac{1}{n}\sum_{i=1}^{n}sgn(x_i - med \; x)sgn(y_i - med \; y),
\end{equation}
\begin{equation}
    q^{*}_{y} = \frac{y_j - y_l}{k_q(n)}, \qquad q^{*}_{x} = \frac{x_j - x_l}{k_q(n)}
\end{equation}
\[
    l = 
		\left\{
		\begin{aligned}
			 \left[n/4\right] + 1& \;\; \text{при} \;\; n/4 \;\; \text{дробном,}\\
			 n/4                  & \;\; \text{при} \;\; n/4 \;\; \text{целом.}\\
		\end{aligned}
		\right.
\]
\[
    j = n - l + 1.
\]
\[
    sgn \; z = 
		\left\{
		\begin{aligned}
			 1 & \; \text{при} \; z > 0,\\
             0 & \; \text{при} \; z = 0,\\
             -1 & \; \text{при} \; z < 0.\\
		\end{aligned}
		\right.
\]
Уравнение регрессии здесь имеет вид
\begin{equation}
    y = \hat{\beta}_{0R} + \hat{\beta}_{1R}x.
\end{equation}

\newpage
\section{Реализация}
Расчёты проводились в среде аналитических вычислений Maxima. Для генерации выборок и создания и отрисовки графиков были использованы библиотечные функции среды разработки. Код скрипта представлен в репозитории на GitHub, ссылка на репозиторий находится в \hyperlink{addition}{\textbf{Приложении А}}.

\newpage
\section{Результаты}
\subsection{Выборочные коэффициенты корреляции}
Ниже, в таблицах \ref{rho0} - \ref{rho09} представлены выборочные коэффициенты корреляции Пирсона, Спирмена и квадрантный коэффициент корреляции для выборок размером 20, 60 и 100 элементов двумерного нормального распределения $N(x, y, 0, 0, 1, 1, \rho)$ с коэффициентами корреляции $\rho = 0, 0.5, 0.9$. 
\begin{table}[h]
    \begin{center}
        \caption{Выборочные коэффициенты корреляции для двумерного нормального распределения, $\rho = 0$}
        \phantom{0}\\
        \begin{tabular}{||c||c|c|c||c|c|c||c|c|c||}\hline
            \multirow{2}*{$\rho = 0$}& \multicolumn{3}{c||}{$r\mathstrut$ (\ref{pierson})} & \multicolumn{3}{c||}{$r_Q$ (\ref{quadrant})} & \multicolumn{3}{c||}{$r_S$ (\ref{spirmen})}\\
            \cline{2-10}
                & $E(z)\mathstrut$ & $E(z^2)$ & $D(z)$ & $E(z)$ & $E(z^2)$ & $D(z)$ & $E(z)$ & $E(z^2)$ & $D(z)$\\
            \hline
            $N=20$ & $0.005\mathstrut$ & 0.053 & 0.053 & 0.006 & 0.053 & 0.053 & 0.007 & 0.052 & 0.052\\
            \hline
            $N=60$ & $0.001\mathstrut$ & 0.018 & 0.018 & 0.004 & 0.017 & 0.017 & 0.004 & 0.018 & 0.017\\
            \hline
            $N=100$ & $0.003\mathstrut$ & 0.011 & 0.011 & 0.001 & 0.01 & 0.01 & 0.004 & 0.011 & 0.011\\
            \hline
        \end{tabular}
    \label{rho0}
    \end{center}
\end{table}

\begin{table}[h]
    \begin{center}
        \caption{Выборочные коэффициенты корреляции для двумерного нормального распределения, $\rho = 0.5$}
        \phantom{0}\\
        \begin{tabular}{||c||c|c|c||c|c|c||c|c|c||}\hline
            \multirow{2}*{$\rho = 0.5$}& \multicolumn{3}{c||}{$r\mathstrut$} & \multicolumn{3}{c||}{$r_Q$} & \multicolumn{3}{c||}{$r_S$}\\
            \cline{2-10}
                & $E(z)\mathstrut$ & $E(z^2)$ & $D(z)$ & $E(z)$ & $E(z^2)$ & $D(z)$ & $E(z)$ & $E(z^2)$ & $D(z)$\\
            \hline
            $N=20$ & $0.5\mathstrut$ & 0.3 & 0.03 & 0.324 & 0.155 & 0.05 & 0.46 & 0.24 & 0.035\\
            \hline
            $N=60$ & $0.5\mathstrut$ & 0.26 & 0.01 & 0.32 & 0.12 & 0.016 & 0.5 & 0.24 & 0.01\\
            \hline
            $N=100$ & $0.5\mathstrut$ & 0.26 & 0.005 & 0.33 & 0.12 & 0.009 & 0.5 & 0.24 & 0.006\\
            \hline
        \end{tabular}
    \label{rho05}
    \end{center}
\end{table}

\begin{table}[h]
    \begin{center}
        \caption{Выборочные коэффициенты корреляции для двумерного нормального распределения, $\rho = 0.9$}
        \phantom{0}\\
        \begin{tabular}{||c||c|c|c||c|c|c||c|c|c||}\hline
            \multirow{2}*{$\rho = 0.9$}& \multicolumn{3}{c||}{$r\mathstrut$} & \multicolumn{3}{c||}{$r_Q$} & \multicolumn{3}{c||}{$r_S$}\\
            \cline{2-10}
                & $E(z)\mathstrut$ & $E(z^2)$ & $D(z)$ & $E(z)$ & $E(z^2)$ & $D(z)$ & $E(z)$ & $E(z^2)$ & $D(z)$\\
            \hline
            $N=20$ & $0.9\mathstrut$ & 0.8 & 0.003 & 0.6916 & 0.5064 & 0.03 & 0.9 & 0.75 & 0.005\\
            \hline
            $N=60$ & $0.89\mathstrut$ & 0.8 & 0.0007 & 0.7 & 0.5 & 0.009 & 0.89 & 0.78 & 0.001\\
            \hline
            $N=100$ & $0.9\mathstrut$ & 0.8 & 0.0004 & 0.7042 & 0.5 & 0.005 & 0.9 & 0.8 & 0.0006\\
            \hline
        \end{tabular}
    \label{rho09}
    \end{center}
\end{table}
Ниже в таблице \ref{mixTable} представлены выборочные коэффициенты корреляции Пирсона, Спирмена и квадрантный коэффициент корреляции для выборок смеси двумерных нормальных распределений (\ref{mix}) размером 20, 60 и 100 элементов.
\newpage
\begin{table}[h]
    \begin{center}
        \caption{Выборочные коэффициенты корреляции для смеси двумерных нормальных распределений}
        \phantom{0}\\
        \begin{tabular}{||c||c|c|c||c|c|c||c|c|c||}\hline
            \multirow{2}*{}& \multicolumn{3}{c||}{$r\mathstrut$} & \multicolumn{3}{c||}{$r_Q$} & \multicolumn{3}{c||}{$r_S$}\\
            \cline{2-10}
                & $E(z)\mathstrut$ & $E(z^2)$ & $D(z)$ & $E(z)$ & $E(z^2)$ & $D(z)$ & $E(z)$ & $E(z^2)$ & $D(z)$\\
            \hline
            $N=20$ & $-0.35\mathstrut$ & 0.57 & 0.45 & 0.5332 & 0.32 & 0.036 & 0.47 & 0.3 & 0.08\\
            \hline
            $N=60$ & $-0.65\mathstrut$ & 0.5 & 0.08 & 0.56 & 0.32 & 0.01 & 0.5 & 0.25 & 0.026\\
            \hline
            $N=100$ & $-0.7\mathstrut$ & 0.52 & 0.03 & 0.56 & 0.32 & 0.007 & 0.47 & 0.23 & 0.02\\
            \hline
        \end{tabular}
    \label{mixTable}
    \end{center}
\end{table}
\subsection{Эллипсы рассеивания}
Ниже на рисунках \ref{rho_0} - \ref{rho_09} представлены эллипсы рассеивания для выборок двумерного нормального распределения размером 20, 60 и 100 элементов, синим цветом обозначены элементы выборок, эллипсы построены согласно формуле (\ref{ellipse}).
\begin{figure}[h]
    \begin{minipage}[h]{0.5\linewidth}
        \center{\includegraphics[width=1\linewidth]{Ris1.png}}
    \end{minipage}
    \begin{minipage}[h]{0.5\linewidth}
        \center{\includegraphics[width=1\linewidth]{Ris2.png}}
    \end{minipage}
    \begin{minipage}[h]{0.5\linewidth}
        \center{\includegraphics[width=1\linewidth]{Ris3.png}}
    \end{minipage}
    \caption{Эллипсы рассеивания для выборок нормального распределения, $\rho = 0$}
    \label{rho_0}
\end{figure}

\begin{figure}[h]
    \begin{minipage}[h]{0.5\linewidth}
        \center{\includegraphics[width=1\linewidth]{Ris4.png}}
    \end{minipage}
    \begin{minipage}[h]{0.5\linewidth}
        \center{\includegraphics[width=1\linewidth]{Ris5.png}}
    \end{minipage}
    \begin{minipage}[h]{0.5\linewidth}
        \center{\includegraphics[width=1\linewidth]{Ris6.png}}
    \end{minipage}
    \caption{Эллипсы рассеивания для выборок нормального распределения, $\rho = 0.5$}
    \label{rho_05}
\end{figure}

\begin{figure}[h!]
    \begin{minipage}[h]{0.5\linewidth}
        \center{\includegraphics[width=1\linewidth]{Ris7.png}}
    \end{minipage}
    \begin{minipage}[h]{0.5\linewidth}
        \center{\includegraphics[width=1\linewidth]{Ris8.png}}
    \end{minipage}
    \begin{minipage}[h]{0.5\linewidth}
        \center{\includegraphics[width=1\linewidth]{Ris9.png}}
    \end{minipage}
    \caption{Эллипсы рассеивания для выборок нормального распределения, $\rho = 0.9$}
    \label{rho_09}
\end{figure}

\newpage
\subsection{Оценки коэффициентов линейной регрессии}
\subsubsection{Выборка без возмущений}
\begin{itemize}
    \item Критерий наименьших квадратов (\ref{mnk1} - \ref{mnk0}):
        \[
            \hat{\beta}_0 \approx 1.903, \quad \hat{\beta}_1 \approx 2.037   
        \]
    \item Критерий наименьших модулей (\ref{mnm1} - \ref{mnm0}):
        \[
            \hat{\beta}_{0R} \approx 2.239, \quad \hat{\beta}_{1R} \approx 1.612
        \]
\end{itemize}
На рисунке \ref{noEmis} представлена выборка без возмущений, график теоретической модели, а также графики, соответствующие линейным регрессиям с коэффициентами, вычисленными согласно методам наименьших квадратов и наименьших модулей.\\
\begin{figure}[h]
    \begin{minipage}[h]{1\linewidth}
        \center{\includegraphics[width=1\linewidth]{Ris10.png}}
    \end{minipage}
    \caption{Выборка без возмущений}
    \label{noEmis}
\end{figure}
\phantom{0}\\
\subsubsection{Выборка с возмущениями}
\begin{itemize}
    \item Критерий наименьших квадратов (\ref{mnk1} - \ref{mnk0}):
        \[
            \hat{\beta}_0 \approx 2.046, \quad \hat{\beta}_1 \approx 0.608
        \]
    \item Критерий наименьших модулей (\ref{mnm1} - \ref{mnm0}):
        \[
            \hat{\beta}_{0R} \approx 2.279, \quad \hat{\beta}_{1R} \approx 1.209
        \]
\end{itemize}
На рисунке \ref{Emis} представлена выборка с возмущениями, график теоретической модели, а также графики, соответствующие линейным регрессиям с коэффициентами, вычисленными согласно методам наименьших квадратов и наименьших модулей.
\begin{figure}[h]
    \begin{minipage}[h]{1\linewidth}
        \center{\includegraphics[width=1\linewidth]{Ris11.png}}
    \end{minipage}
    \caption{Выборка с возмущениями}
    \label{Emis}
\end{figure}

\newpage
\section*{Заключение}
\addcontentsline{toc}{section}{Заключение}
В ходе выполнения лабораторной работы были построены выборки двумерного нормального распределения с коэффициентами корреляции $\rho = 0, 0.5, 0.9$ и выборки смеси нормальных распределений. Исходя из оценок выборочных коэффициентов корреляции можно сделать вывод, что квадрантный выборочный коэффициент имеет наибольшее отклонение от теоретического значения коэффициента корреляции, но с увеличением мощности выборки все выборочные коэффициенты корреляции стремятся к своим теоретическим значениям.\\
\noindentИз графиков двумерных нормальных распределений видно, что чем больше коэффициент корреляции, тем больше сужается эллипс рассеивания, в пределе вырождаясь в прямую, при модуле коэффициента корреляции равном единице.\\
Также в ходе выполнения работы были построены линейные регрессии для нормально распределенной выборки. Коэффициенты регрессии вычислялись с помощью методов наименьших квадратов и наименьших модулей.\\
Исходя из полученных результатов можно сделать вывод о том, что метод наименьших квадратов позволяет более точно определить коэффициенты регрессии, однако этот метод не устойчив к возмущениям в выборке. С другой стороны метод наименьших модулей даёт менее точную оценку, однако, являясь робастным, он гораздо устойчивее к возмущениям и хорошо подходит для работы с выборками, имеющими сильные отклонения в отдельных точках.\\

\newpage
\addcontentsline{toc}{section}{Список Литературы}
\begin{thebibliography}{9}

	\bibitem{theory}
	Теоретическое приложение к лабораторным работам №5-8 по дисциплине «Математическая статистика». -- СПб.: СПбПУ, 2020. -- 22 c 

\end{thebibliography}
\newpage
\section*{\hypertarget{addition}{Приложение А. Репозиторий с исходным кодом}}
\addcontentsline{toc}{section}{Приложение А. Репозиторий с исходным кодом}
Ссылка на репозиторий GitHub с исходным кодом: \url{https://github.com/pchn/TeorVer}
\end{document}
\end{document}