\documentclass[12pt]{article}
\usepackage[utf8]{inputenc}
\usepackage[russian]{babel}
\usepackage{pscyr}
\usepackage[T2A]{fontenc}
\usepackage{geometry}
\usepackage{graphicx}
\usepackage{multirow}
%\usepackage{hhline}
\usepackage{amsmath}
\usepackage{amssymb}
\usepackage[unicode, pdftex]{hyperref}
\usepackage{xcolor}

\DeclareUnicodeCharacter{2212}{}
\geometry {	
	a4paper, 
	left   = 20mm, 
	right  = 20mm, 
	top    = 20mm, 
	bottom = 20mm
}

\definecolor{urlcolor}{HTML}{000000} 
\definecolor{linkcolor}{HTML}{000000}

\graphicspath{{resource/}}

\begin{document}

\begin{titlepage}
	\begin{center}
		\hfill \break
		{\textbf{Санкт-Петербургский политехнический университет Петра Великого}}\\
		\hfill \break
		\textbf{Институт прикладной математики и механики}\\
		\hfill \break
		\textbf{Кафедра <<Телематика (при ЦНИИ РТК)>>}\\
		\vfill
		\large{\bfseries Отчет по лабораторным работам № 7, 8}\\
		\hfill \break
		\hfill \break
		\hfill \break
		\hfill \break
		По дисциплине <<Теория вероятностей и Математическая статистика>>\\
		\hfill \break
		\hfill \break
		\hfill \break
	\end{center}
 
	\normalsize
	{ 
		\begin{tabular}{lp{2cm}cr}
			Выполнил &&&\\
			Студент гр. 3630201/80101&&\underline{\hspace{1.5cm}}& Печеный Н. А.\\\\
			Руководитель&&&\\ 
			к.ф.-м.н., доцент && \underline{\hspace{1.5cm}}& Баженов А. Н. \\\\
			&&&<<\underline{\phantom{333}}>>\underline{\phantom{сентября000}}
			2021г.
		\end{tabular}
	}
\vfill

\begin{center} Санкт-Петербург \\2021 \end{center}
\end{titlepage}

\newpage

\setcounter{page}{2}

\setlength{\parindent}{1cm}

\tableofcontents

\newpage

\listoftables

\newpage

\section{Постановка задачи}
\begin{enumerate}
    \item  Сгенерировать выборку объёмом 100 элементов для нормального распределения $N(x,0,1)$. По сгенерированной выборке оценить параметры $\mu$ и $\sigma$ нормального закона методом максимального правдоподобия. В качестве основной гипотезы $H_0$ будем считать, что сгенерированное распределение имеет вид $N(x, \hat{\mu}, \hat{\sigma})$. Проверить основную гипотезу, используя критерий согласия $\chi^2$.  В качестве уровня значимости взять $\alpha = 0.05$. Привести таблицу вычислений $\chi^2$.\\
    \noindentИсследовать точность (чувствительность) критерия $\chi^2$ --- сгенерировать выборки равномерного распределения и распределения Лапласа малого объема (например, 20 элементов). Проверить их на нормальность.
    \item Для двух выборок размерами 20 и 100 элементов, сгенерированных согласно нормальному закону $N(x,0,1)$, для параметров положения и масштаба построить асимптотически нормальные интервальные оценки на основе точечных оценок метода максимального правдоподобия и классические интервальные оценки на основе статистик $\chi^2$ и Стьюдента. В качестве параметра надёжности взять $\gamma = 0.95$
\end{enumerate}

\newpage
\section{Теория}
\subsection{Метод максимального правдоподобия}
$L(x_1,\dots,x_n,\theta)$ --- функция правдоподобия (ФП), рассматриваемая как функция неизвестного параметра $\theta$:
\begin{equation}
    L(x_1,\dots,x_n,\theta) = f(x_1,\theta)\dots f(x_n, \theta).
\end{equation}
Оценка максимального правдоподобия:
\begin{equation}
    \hat{\theta}_{\text{МП}} = \arg\max L(x_1,\dots,x_n,\theta)
\end{equation}
Система уравнений правдоподобия (в случае дифференцируемости функции правдоподобия):
\begin{equation}
    \frac{\partial L}{\partial \theta_k} = 0 \; \text{ или } \; \frac{\partial \ln L}{\partial \theta_k} = 0, \; k = 1,\dots,m.
    \label{difEquation}
\end{equation}

\subsection{Проверка гипотезы о законе распределения генеральной совокупности. Метод хи-квадрат}
Выдвинута гипотеза $H_0$ о генеральном законе распределения с функцией
распределения $F(x)$.\\
\phantom{0}\\
Рассматриваем случай, когда гипотетическая функция распределения $F(x)$
не содержит неизвестных параметров.\\
\phantom{0}\\
\textbf{Правило проверки гипотезы о законе распределения по методу $\chi^2$.}
\begin{enumerate}
    \item Выбираем уровень значимости $\alpha$.

    \item По таблице [2, с. 358] находим квантиль $\chi^2_{1-\alpha}(k-1)$ распределения хи-квадрат с $k-1$ степенями свободы порядка $1-\alpha$.

    \item С помощью гипотетической функции распределения $F(x)$ вычисляем
    вероятности $p_i = P(X\in \Delta_i), i=1,\dots,k$.

    \item Находим частоты $n_i$ попадания элементов выборки в подмножества
    $\Delta_i, i = 1,\dots,k$.

    \item Вычисляем выборочное значение статистики критерия $\chi^2$ :
    \[
        \chi^2_B = \sum_{i=1}^{n}\frac{(n_i - np_i)^2}{np_i}.    
    \]

    \item Сравниваем $\chi^2_B$ и квантиль  $\chi^2_{1-\alpha}(k-1)$.
    \begin{itemize}
        \item[$\text{а)}$] Если $\chi^2_B < \chi^2_{1-\alpha}(k-1)$, то гипотеза $H_0$ на данном этапе проверки принимается.
        \item[$\text{б)}$] Если $\chi^2_B \geq \chi^2_{1-\alpha}(k-1)$, то гипотеза $H_0$ отвергается, выбирается одно из альтернативных распределений, и процедура проверки повторяется.
    \end{itemize}
\end{enumerate}
Количество интервалов $k$ можно определить с помощью эвристики:
\begin{equation}
    k \approx 1.72\cdot\sqrt[3]{n}
    \label{k}
\end{equation}
\subsection{Доверительные интервалы}
Дана выборка размером $n (x_1,\dots,x_n)$ из генеральной совокупности. Для нее построим выборочное среднее $\overline{x}$ и среднеквадратическое отклонение $s$.\\
\phantom{0}\\
Параметры расположения $\mu$ и масштаба $\sigma$ неизвестны. Построим для них доверительный интервал с доверительной вероятностью $\gamma$.

\subsubsection{Оценка на основе статистики Стьюдента и хи-квадрат}
Оценка для параметра положения:
\begin{equation}
    P\left(\overline{x}-\frac{s\cdot t_{1-\alpha/2}(n-1)}{\sqrt{n-1}} < \mu < \overline{x}+\frac{s\cdot t_{1-\alpha/2}(n-1)}{\sqrt{n-1}}\right) = \gamma,
    \label{mStud}
\end{equation}
где $1 - \alpha = \gamma, \; t_{1-\alpha/2}(n-1)$ --- квантиль распределения Стьюдента с $(n-1)$ степенями свободы порядка $1 - \alpha/2$.\\
\phantom{0}\\
Оценка для параметра масштаба:
\begin{equation}
    P\left(\frac{s\sqrt{n}}{\sqrt{\mathstrut \chi^2_{1 - \alpha/2}(n-1)}} < \sigma < \frac{s\sqrt{n}}{\sqrt{\mathstrut \chi^2_{\alpha/2}(n-1)}}\right) = \gamma,
    \label{sStud}
\end{equation}
где $1 - \alpha = \gamma, \; \chi^2_{p}(n-1)$ --- квантиль распределения хи-квадрат с $(n-1)$ степенями свободы порядка $p$.\\
\phantom{0}\\
Эти оценки справедливы для выборки из нормальной генеральной совокупности.

\subsubsection{Асимптотические оценки на основе центральной предельной теоремы}
Оценка для параметра положения:
\begin{equation}
    P\left(\overline{x}-\frac{s\cdot u_{1-\alpha/2}}{\sqrt{n}} < \mu < \overline{x}+\frac{s\cdot u_{1-\alpha/2}}{\sqrt{n}}\right) \approx \gamma,
    \label{mAsympt}
\end{equation}
где $1-\alpha = \gamma, u_{1-\alpha/2}$ --- квантиль стандартного нормального распределения порядка $1 - \alpha/2$.\\
\phantom{0}\\
Для оценки параметра масштаба необходимо рассчитать выборочный эксцесс $e = \frac{m_4}{s^4}-3$, где $m_4 = \frac{1}{n}\sum(x_i - \overline{x})^4$ --- четвёртый выборочный центральный момент.\\
\phantom{0}\\
Парметр масштаба можно оценить так:
\begin{equation}
    P\left(s(1 + U)^{-0.5} < \sigma < s(1 - U)^{-0.5}\right) \approx \gamma,
    \label{sAsympt}
\end{equation} 
где $U = u_{1-\alpha/2}\sqrt{\mathstrut (e+2)/n}, \; u_{1-\alpha/2}$ --- квантиль стандартного нормального распределения порядка $1 - \alpha/2$.\\
\phantom{0}\\
Эти оценки справедливы для выборки из генеральной совокупности, которая имеет конечные центральные моменты вплоть до 4 порядка и конечное матожидание.

\section{Реализация}
Расчёты проводились в среде аналитических вычислений Maxima. Для генерации выборок и создания и отрисовки графиков были использованы библиотечные функции среды разработки. Код скрипта представлен в репозитории на GitHub, ссылка на репозиторий находится в \hyperlink{addition}{\textbf{Приложении А}}.\\
\phantom{0}\\
Можно показать, что для 
\[
    f(x,\mu,\sigma) = \frac{1}{\sigma\sqrt{2\pi}}e^{-frac{(x-\mu)^2}{2\sigma^2}}    
\]
решениями системы (\ref{difEquation}) являются значения $\mu = \overline{x}, \sigma = \sqrt{\overline{x^2} - \overline{x}^2}$ \cite{site}. Именно таким образом искались оценки для параметров $\mu$ и $\sigma$ методом максимального правдоподобия.
\newpage
\section{Результаты}
\subsection{Проверка гипотезы о законе распределения генеральной совокупности. Метод хи-квадрат}
Метод максимального правдоподобия:
\[
    \hat{\mu} \approx 0.03 , \quad \hat{\sigma} \approx 1.11    
\]
Критерий согласия $\chi^2$:\\
\begin{itemize}
    \item Количество промежутков $k=8$ (\ref{k})
    \item Уровень значимости $\alpha = 0.05$
    \item Тогда квантиль из таблицы [2, c.358] $\chi^2_{1-\alpha}(k-1) = \chi^2_{0.95}(7) = 14.0671$
\end{itemize}
В таблице \ref{chinorm} приведено вычисление $\chi^2_B$ при проверке гипотезы $H_0$ о нормальном законе распределения $N(x, \hat{\mu}, \hat{\sigma})$.\\
\begin{table}[h!]
    \begin{center}
        \caption{Вычисление $\chi^2_B$ при проверке гипотезы $H_0$ о нормальном законе распределения $N(x, \hat{\mu}, \hat{\sigma})$}
        \phantom{0}\\
        \begin{tabular}{|c|c|c|c|c|c|c|}\hline
            $i$ & $\Delta_i$ & $n_i$ & $p_i$ & $np_i$ & $n_i - np_i$ & $\frac{(n_i-np_i)^2}{np_i}$\\
            \hline
            $1$ & $-\infty,-1.352$ & 8 & 0.088 & 8.8 & -0.8 & 0.0727\\
            \hline
            $2$ & $-1.352, -0.905$ & 9 & 0.095 & 9.5 & -0.5 & 0.0263\\
            \hline
            $3$ & $-0.905, -0.458$ & 14 & 0.141 & 14.1 & -0.1 & 0.0007\\
            \hline
            $4$ & $-0.458, -0.011$ & 19 & 0.172 & 17.2 & 1.8 & 0.1884\\
            \hline
            $5$ & $-0.011, 0,437$ & 17 & 0.173 & 17.3 & -0.3 & 0.0052\\
            \hline
            $6$ & $0.437, 0.884$ & 14 & 0.143 & 14.3 & -0.3 & 0.0063\\
            \hline
            $7$ & $0.884, 1.331$ & 11 & 0.097 & 9.7 & 1.3 & 0.1742\\
            \hline
            $8$ & $1,331, +\infty$ & 8 & 0.091 & 9.1 & -1.1 & 0.1329\\
            \hline
            $\sum$ & --- & 100 & 1.000 & 100.0 & 0.0 & $0.6067 = \chi^2_B$\\
            \hline
        \end{tabular}
    \label{chinorm}
    \end{center}
\end{table}
\phantom{0}\\
Табличное значение $\chi^2_{0.95}(7) = 14.0671$, выборочное $\chi^2_B = 0.6067$, это значит, что на данном этапе гипотезу $H_0$ можно принять.\\
\phantom{0}\\
\textbf{Исследование на чувствительность.}\\
\phantom{0}\\
\textbf{Распределение Лапласа}.\\
Для выборок из 20 элементов был выбран $k=5$. Уровень значимости был выбран $\alpha = 0.05$. В таблице \ref{chiLap} вычислен выборочный коэффициент $\chi^2_B$ для выборки, cоответствующей распределению Лапласа $L(x,0,1)$
\newpage  
\begin{table}[h!]
    \begin{center}
        \caption{Вычисление $\chi^2_B$ при проверке гипотезы о нормальном законе распределения для выборки $L(x,0,1)$}
        \phantom{0}\\
        \begin{tabular}{|c|c|c|c|c|c|c|}\hline
            $i$ & $\Delta_i$ & $n_i$ & $p_i$ & $np_i$ & $n_i - np_i$ & $\frac{(n_i-np_i)^2}{np_i}$\\
            \hline
            $1$ & $-\infty, -0.915$ & 1 & 0.2002 & 4.004 & -3.004 & 2.2538\\
            \hline
            $2$ & $-0.915, 0.054$ & 5 & 0.3259 & 6.518 & -1.518 & 0.3535\\
            \hline
            $3$ & $0.054, 1.022$ & 9 & 0.2939 & 5.878 & 3.122 & 1.6582\\
            \hline
            $4$ & $1.022, 1.991$ & 4 & 0.1116 & 2.232 & 1.768 & 1.4004\\
            \hline
            $5$ & $1.991, +\infty$ & 1 & 0.0684 & 1.368 & -0.368 & 0.0989\\
            \hline
            $\sum$ & --- & 20 & 1.0000 & 20.000 & 0.000 & $5.7648 = \chi^2_B$\\
            \hline
        \end{tabular}
    \label{chiLap}
    \end{center}
\end{table}
Табличное значение $\chi^2_{0.95}(4) = 9.4877$, выборочное значение $\chi^2_B = 5.7648$, значит гипотеза также может быть принята на данном этапе.\\
\phantom{0}\\
\textbf{Равномерное распределение.}\\
В таблице \ref{chiUni} вычислен выборочный коэффициент $\chi^2_B$ для выборки, соответствующей равномерному распределению.\\
\begin{table}[h!]
    \begin{center}
        \caption{Вычисление $\chi^2_B$ при проверке гипотезы о нормальном законе распределения для выборки $U(x,-\sqrt{3},\sqrt{3})$}
        \phantom{0}\\
        \begin{tabular}{|c|c|c|c|c|c|c|}\hline
            $i$ & $\Delta_i$ & $n_i$ & $p_i$ & $np_i$ & $n_i - np_i$ & $\frac{(n_i-np_i)^2}{np_i}$\\
            \hline
            $1$ & $-\infty, -1.402$ & 1 & 0.0952 & 1.904 & -0.904 & 0.4292\\
            \hline
            $2$ & $-1.402, -0.406$ & 4 & 0.2875 & 5.750 & -1.750 & 0.5326\\
            \hline
            $3$ & $-0.406, 0.589$ & 6 & 0.2875 & 5.750 & 0.250 & 0.0109\\
            \hline
            $4$ & $0.589, 1.586$ & 8 & 0.2875 & 5.750 & 2.250 & 0.8804\\
            \hline
            $5$ & $1.586, +\infty$ & 1 & 0.0423 & 0.846 & 0.154 & 0.0280\\
            \hline
            $\sum$ & --- & 20 & 1.0000 & 20.000 & 0.000 & $1.8811 = \chi^2_B$\\
            \hline
        \end{tabular}
    \label{chiUni}
    \end{center}
\end{table}
\phantom{0}\\
Табличное значение $\chi^2_{0.95}(4) = 9.4877$, выборочное значение $\chi^2_B = 1.8811$, значит гипотеза также может быть принята на данном этапе.\\
\subsection{Доверительные интервалы для параметров нормального распределения}
В таблице \ref{tabStud} представлены оценки на основе статистик Стьюдента и хи-квадрат.\\
\newpage
\begin{table}[h!]
    \begin{center}
        \caption{Интервальные оценки на основе статистик Стьюдента и хи-квадрат}
        \phantom{0}\\
        \begin{tabular}{|c|c|c|}\hline
            $n = 20$ & $m$ (\ref{mStud}) & $\sigma$ (\ref{sStud})\\
            \hline
            & $-0.43 < m < 0.63$ & $0.86 < \sigma < 1.65$\\
            \hline
            &&\\
            \hline
            $n=100$ & $m$ & $\sigma$\\
            \hline
            & $-0.22 < m < 0.15$ & $0.82 < \sigma < 1.08$\\
            \hline
        \end{tabular}
    \label{tabStud}
    \end{center}
\end{table}
\subsection{Доверительные интервалы для параметров произвольного распределения. Асимптотический подход}
В таблице \ref{tabAsympt} представлены асимптотические интервальные оценки.\\
\begin{table}[h!]
    \begin{center}
        \caption{Асимптотические интервальные оценки}
        \phantom{0}\\
        \begin{tabular}{|c|c|c|}\hline
            $n = 20$ & $m$ (\ref{mAsympt}) & $\sigma$ (\ref{sAsympt})\\
            \hline
            & $-0.38 < m < 0.58$ & $0.91 < \sigma < 1.49$\\
            \hline
            &&\\
            \hline
            $n=100$ & $m$ & $\sigma$\\
            \hline
            & $-0.21 < m < 0.15$ & $0.81 < \sigma < 1.08$\\
            \hline
        \end{tabular}
    \label{tabAsympt}
    \end{center}
\end{table}
\newpage
\section*{Заключение}
\addcontentsline{toc}{section}{Заключение}
В рамках лабораторной работы была сгенерирована выборка размером 100 элементов для нормального распределения $N(x,0,1)$. Для неё методом максимального правдоподобия были найдены оценки параметров $\mu$ и $\sigma$. Также для данной функции гипотеза о её нормальности была проверена критерием $\chi^2$. Также критерием $\chi^2$ были проверены гипотезы о нормальности выборок из 20 элементов, соответствующих распределению Лапласа и равномерному распределению.\\
\phantom{0}\\
Было установлено, что применение метода максимального правдоподобия для нормального распределения сводится к нахождению выборочного среднего и среднеквадратического отклонения.\\
\phantom{0}\\
Критерий $\chi^2$ не отверг гипотезу о соответствии нормальной выборки нормальному распределению, чего и следовало ожидать.\\
\phantom{0}\\
Критерий $\chi^2$ также не отверг гипотезу о нормальности выборок, соответствующих равномерному распределению и распределению Лапласа размером 20 элементов. Это объясняется тем, что критерий $\chi^2$ носит асимптотический характер, а значит может давать ошибки при малых размерах выборки.\\
\phantom{0}\\
Также для выборок нормального распределения размером 20 и 100 элементов были построены доверительные интервалы для парамметров положения и масштаба. Интервалы были получены двумя способами: с помощью классических интервальных оценок и асимптотических оценок.\\
\phantom{0}\\
На выборках малого размера асимптотические оценки оказались точнее, что странно, но может быть списано но ошибки, возникающие при работе с выборками малого размера. При выборках большего размера разницы в доверительных интервалах почти не наблюдается, что может свидетельствовать о том, что при увеличении мощности выборки ещё сильнее можно будет наблюдать более высокую точность классических интервальных оценок.\\
\phantom{0}\\


\newpage
\addcontentsline{toc}{section}{Список Литературы}
\begin{thebibliography}{9}

	\bibitem{theory}
    Теоретическое приложение к лабораторным работам №5-8 по дисциплине «Математическая статистика». -- СПб.: СПбПУ, 2020. -- 22 c 
    
    \bibitem{maks}
    Максимов Ю.Д. Математика. Теория и практика по математической статистике. Конспект-справочник по теории вероятностей : учеб. пособие / Ю.Д. Максимов; под ред. В.И. Антонова. — СПб. : Изд-во Политехн. ун-та, 2009. — 395 с. (Математика в политехническом университете).

    \bibitem{site}
    Методы нахождения оценок: метод максимального правдоподобия - \url{https://nsu.ru/mmf/tvims/chernova/ms/lec/node14.html} (дата обращения - 09.01.2021)

\end{thebibliography}
\newpage
\section*{\hypertarget{addition}{Приложение А. Репозиторий с исходным кодом}}
\addcontentsline{toc}{section}{Приложение А. Репозиторий с исходным кодом}
Ссылка на репозиторий GitHub с исходным кодом: \url{https://github.com/pchn/TeorVer}
\end{document}